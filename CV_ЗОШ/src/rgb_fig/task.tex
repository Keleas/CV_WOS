\begin{problem}{Параметры геометрических фигур}
{\textsl{standard input}}{\textsl{standard output}}
{30 секунд}{256 мебибайт}{}

Первой задачей по машинному зрению будет работа с геометрическими фигурами. 
Ваша задача - написать программу, которая посчитает радиус круга, сторону квадрата и катет равнобедренного прямоугольного треугольника, изображенных на картинке. 

Важно, что в данной задаче цвет каждой геометрической фигуры остается неизменным на всех изображениях.

Для работы с изображенями можете использовать слудующуйю инстркуцию:
new_data = np.loadtxt('test_image_00.txt')
new_data = new_data.reshape((N, M, K))
где N,M,K - размер исходного изображения, который указан в первой строчке test_image_00.txt.

\InputFile

В первой строке введите строку - имя изображения с его расширением $example.jpg$

Примеры изображений для теста Вы можете скачать в приложении.

\OutputFile

В каждой отдельной строке строке выведите числа - радиус круга, сторону квадрата, катет треугольнка"--- ответ на задачу с точностью до сотых.
Используйте следующий код: print("\{:.2f\}".format(s)) , где s - результат.

\Examples

\begin{example}
\exmp{
test_image_00.jpg
}{%
113.42
347.00
407.46
}%

\exmp{
test_image_01.jpg
}{%
156.72
234.00
413.20
}%


\end{example}

\Explanations

\end{problem}
