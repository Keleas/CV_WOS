\begin{problem}{Пустыня или лес?}
{\textsl{standard input}}{\textsl{standard output}}
{30 секунд}{256 мебибайт}{}

Часто при анализе изображений местности необходимо понять ее характер. В частности, если определить, что на изображении преобладет вода, то имеет смысл искать корабли на таком изображении. Если на картинке густой лес, то, возможно, это не лучшая зона для посадки дрона или беспилотника.

Ваша задача - написать программу, которая будет отличать лес от пустыни. В приложении можно найти реальные спутниковые снимки лесов и пустынь. Подобные примеры изображений присутствуют в тесте.   

Для работы с изображенями можете использовать слудующуйю инстркуцию:
new_data = np.loadtxt('test_image_00.txt')
new_data = new_data.reshape((N, M, K))
где N,M,K - размер исходного изображения, который указан в первой строчке test_image_00.txt.

\InputFile

В первой строке введите строку - имя изображения с его расширением $example.jpg$

Примеры изображений для теста Вы можете скачать в приложении.

\OutputFile

В единственной строке выведите одно слово "--- ответ на задачу.

\Examples

\begin{example}
\exmp{
test_image_01.jpg
}{%
Forest
}%

\exmp{
test_image_06.jpg
}{%
Desert
}%


\end{example}

\Explanations

\end{problem}
