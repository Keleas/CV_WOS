\begin{problem}{Площадь корабля}
{\textsl{standard input}}{\textsl{standard output}}
{30 секунд}{256 мебибайт}{}

В продолжении темы предыдущей задачи. Допустим, нам удалось понять, что на рассматриваемом изображении находится море. Значит, есть шанс на то, что можно найти корабли на картинке. 

Ваша задача - написать программу, которая посчитает площадь одного или нескольких кораблей на входном изображении. Площадь найти в пикселях. В приложении можно найти реальные спутниковые снимки кораблей. Подобные примеры изображений присутствуют в тесте.

Важно, что ответ в данной задаче лежит в небольшом интервале. Поэтому тест будет засчитан, если ваше решение попадет в заданый интервал.

Для работы с изображенями можете использовать слудующуйю инстркуцию:
new_data = np.loadtxt('test_image_00.txt')
new_data = new_data.reshape((N, M, K))
где N,M,K - размер исходного изображения, который указан в первой строчке test_image_00.txt.

\InputFile

В первой строке введите строку - имя изображения с его расширением $example.jpg$

Примеры изображений для теста Вы можете скачать в приложении.

\OutputFile

В единственной строке выведите одно число "--- ответ на задачу.

\Examples

\begin{example}
\exmp{
test_image_01.jpg
}{%
6733.0
}%

\exmp{
test_image_06.jpg
}{%
0.0
}%

\exmp{
test_image_08.jpg
}{%
7183.0
}%

\end{example}

\Explanations

\end{problem}
